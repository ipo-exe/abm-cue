\documentclass{article}
\usepackage[utf8]{inputenc}
\usepackage{amsmath}
\usepackage{amssymb}
\title{CUE - Cooperation in Urban Environments: Theoretical Notation}
\author{}
\begin{document}

\maketitle

\section{Model general notation}

\begin{itemize}
    \item $A$ -- Agent;
    \item $P$ -- Place;
    \item $\sigma$ -- trait value;
    \item $D$ -- critical Hamming distance of trait;
    \item $R$ -- critical radius;
    \item $C$ -- openness to contamination, or degree of interaction influence;
    \item $M$ -- memory size;

\end{itemize}

\section{Simulation notation}

\begin{itemize}
    \item $t$ -- Timestep;
    \item $i$ -- Agent ID: unique integer number;
    \item $j$ -- Place ID: unique integer number;
    \item $T$ -- Total number of timesteps;
    \item $x$ -- position coordinate;
    \item $y$ -- position coordinate;
\end{itemize}

\section{Agents}

\par Agents are denoted by $A$. Any specific Agent is denoted by $A^{i}$, where $i$ is the ID number of the Agent in a world of $N_A$ Agents.

\subsection{Parameters of Agents}

\par Parameters are those static, constant attributes. They can be accessed by their names in the Agent subscript:

\begin{itemize}
    \item $A^{i}_{D}$ -- $D$ of $A^{i}$;
    \item $A^{i}_{R}$ -- $R$ of $A^{i}$;
    \item $A^{i}_{C}$ -- $C$ of $A^{i}$;
    \item $A^{i}_{M}$ -- $M$ of $A^{i}$;
\end{itemize}

\subsection{Variables of Agents}

\par Variables are those mutable, changing attributes. They can be accessed by their names in the Agent subscript and by time step $t$ in the Agent superscript:

\begin{itemize}
    \item $A^{i}_{\sigma}(t)$ -- $\sigma$ of $A^{i}$ in time step $t$;
    \item $A^{i}_{x}(t)$ -- $x$ position coordinate of $A^{i}$ in time step $t$;
    \item $A^{i}_{y}(t)$ -- $y$ position coordinate of $A^{i}$ in time step $t$;
\end{itemize}


\section{Places}

\par Places are denoted by $P$. Any specific Agent is denoted by $P^{j}$, where $j$ is the ID number of the Place in a world of $N_P$ Places. Any specific Place $j$ is also denoted by it's own unique position: $P_{x,y}$ in 2D or $P_{x}$ in 1D.

\subsection{Parameters of Places}

\par Parameters are those static, constant attributes. They can be accessed by their names in the Agent subscript:

\begin{itemize}
    \item $P^{j}_{C}$ -- $C$ of $P^{j}$;
    \item $P^{j}_{x}$ -- $x$ position coordinate of $P^{j}$;
    \item $P^{j}_{y}$ -- $y$ position coordinate of $P^{j}$;
\end{itemize}

\subsection{Variables of Places}

\par Variables are those mutable, changing attributes. They can be accessed by their names in the Place subscript and by time step $t$ in the Place superscript:

\begin{itemize}
    \item $P^{j}_{\sigma}(t)$ -- $\sigma$ of $P^{j}$ in time step $t$;
\end{itemize}

\section{Model equations}

\subsection{Interaction equations}

\par When Agents interact with Places, both get some contamination from each other. Their next $t+1$ value of $\sigma$ changes by the following equations. For Agents:
\begin{equation}
    A^{i}_{\sigma}(t + 1) = \frac{A^{i}_{\bar{\sigma}}(t) + P^{j}_{\sigma}(t) \cdot A^{i}_{C}}{1 + A^{i}_{C}} \quad \forall i, t
\end{equation}
And for Places:
\begin{equation}
    P^{j}_{\sigma}(t + 1) = \frac{P^{j}_{\sigma}(t) + A^{i}_{\bar{\sigma}}(t)\cdot P^{j}_{C}}{1 + P^{j}_{C}} \quad \forall i, j, t
\end{equation}
Where $P^{j}_{x} = A^{i}_{x}(t)$ and $P^{j}_{x} = A^{i}_{y}(t)$. The value of $A^{i, t}_{\bar{\sigma}}$ is a function of the Agent's previous traits ($A_{\sigma}$) and memory size ($A_M$):
\begin{equation}
    A^{i}_{\bar{\sigma}}(t) = \Psi(A^{i}_{\sigma}, A^{i}_{M}) \quad \forall i, t
\end{equation}

Currently, the function $\Psi$ the average of $\sigma$ values allocated in the Agent's memory:
\begin{equation}
    A^{i}_{\bar{\sigma}}(t) = \frac{1}{A^{i}_{M}} \sum_{n=0}^{A^{i}_{M}} A^{i}_{\sigma}(t - n) \quad \forall i, t
\end{equation}

\subsection{Random walk}

\subsubsection{Set of candidate Places}

\par At any time step, each Agent has a set $\mathbb{P}$ of candidate Places to move in so they can interact. This set is made of Places $P$ within the Agent's window of sight of size $A_{R}$ and below the Agent's interaction threshold $A_{D}$.

\par In the case of considering \textbf{euclidean} distances between positions:
\begin{equation}
    P \in \mathbb{P} \;\big|\;|P_{x} - A_{x}| \leq A_{R} \cap |P_{y} - A_{y}| \leq A_{R} \cap |P_{\sigma} - A_{\bar{\sigma}}| \leq A_{D}
\end{equation}

\par In the case of considering \textbf{non-euclidean} distances between positions:
\begin{equation}
    P \in \mathbb{P} \;\big|\; \Phi(A_{x,y},P_{x,y}) \leq A_{R} \cap |P_{\sigma} - A_{\bar{\sigma}}| \leq A_{D}
\end{equation}
Where $\Phi(A_{x,y},P_{x,y})$ is a function that returns the \textbf{path distance} from position $A_{x,y}$ to $P_{x,y}$.

\subsubsection{Likelihood weight}

If the number of Places in the set $\mathbb{P}$ of candidate is highet than 1, then Agents have to choose which Place to go in the next time step. This choice is basically random. However, the \textit{likelihood weight} of a any candidate Place to be chosen is the sum of two components:
\begin{enumerate}

\item The trait of the Place: Agents have a tendency to go to Places like them;
\item The distance of the Place: Agents have a tendency to go to Places closer.
\end{enumerate}
Hence:
\begin{equation}
    P^{j}_L = f(P_{\sigma}, A_{\sigma}) + h(P_{x,y}, A_{x,y})
\end{equation}
Where $P^{j}_L$ is the likelihood of the candidate Place $P^{j}$; $f(P_{\sigma}, A_{\sigma})$ is the trait component, and; $h(P_{x,y}, A_{x,y})$ is the distance component. This value is turned into a sampling probability by normalization (divison by the sum of all places).

\subsubsection{Uniform weighting function}

\par The \texttt{uniform} weighting function is defined to set all available places the same likelihood to be chosen by a given Agent. Therefore:
\begin{equation}
    P^{j}_L = \frac{1}{{\mathbb{P}_N}}
\end{equation}
Where $P^{j}_L$ is the likelihood of the candidate Place $P^{j}$ and ${\mathbb{P}_N}$ is the number of candidate Places in $\mathbb{P}$.

\subsubsection{Linear weighting function}

\par The \texttt{linear} weighting function is defined to set the likelihoods of candidate Places proportinal to the $\sigma$ discrepancy or the $\Phi(A_{x,y},P_{x,y})$ distance. In the case of trait component:
\begin{equation}
     f(P_{\sigma}, A_{\sigma}) = 1 - \frac{|A_{\sigma} - P^{j}_{\sigma}|}{\sum_{j=0}^{\mathbb{P}_N} |A_{\sigma} - P^{j}_{\sigma}|}
\end{equation}
Where $P^{j}_L$ is the likelihood of the candidate Place $P^{j}$ and ${\mathbb{P}_N}$ is the number of candidate Places in $\mathbb{P}$. In the case of distance component:
\begin{equation}
      h(P_{x,y}, A_{x,y}) = 1 - \frac{\Phi(A_{x,y},P^{j}_{x,y})}{\sum_{j=0}^{\mathbb{P}_N} \Phi(A_{x,y},P^{j}_{x,y})}
\end{equation}
Where $P^{j}_L$ is the likelihood of the candidate Place $P^{j}$ and ${\mathbb{P}_N}$ is the number of candidate Places in $\mathbb{P}$.

\subsubsection{Exponential weighting function}

\par The \texttt{exponential} weighting function is defined to set the likelihoods of candidate Places to decay with the $\sigma$ discrepancy or the $\Phi(A_{x,y},P_{x,y})$ distance. In the case of trait component:
\begin{equation}
     f(P_{\sigma}, A_{\sigma}) = e^{- \alpha * |A_{\sigma} - P^{j}_{\sigma}|}
\end{equation}
In the case of distance component:
\begin{equation}
      h(P_{x,y}, A_{x,y}) = e^{- \alpha * \Phi(A_{x,y},P^{j}_{x,y})}
\end{equation}
By default, the $\alpha$ parameter is set to 1.

\subsubsection{Gravity weighting function}

\par The \texttt{gravity} weighting function is defined to set the likelihoods of candidate Places to decay with the $\sigma$ discrepancy or the $\Phi(A_{x,y},P_{x,y})$ distance. In the case of trait component:
\begin{equation}
     f(P_{\sigma}, A_{\sigma}) = \frac{1}{|A_{\sigma} - P^{j}_{\sigma}|^{\gamma}}
\end{equation}
In the case of distance component:
\begin{equation}
      h(P_{x,y}, A_{x,y}) = \frac{1}{\Phi(A_{x,y},P^{j}_{x,y})^{\gamma}}
\end{equation}
By default, the $\gamma$ parameter is set to 1.


\subsubsection{Exception cases}

\par When the set $\mathbb{P}$ of candidate Places is empty so ${\mathbb{P}}_N = 0$ then there is no interaction. This Agent will remain quiet until the suroundings Places yield an non-empty set $\mathbb{P}$.

\par When the set $\mathbb{P}$ of candidate Places holds only one candidade place so ${\mathbb{P}}_N = 1$ then the likelihood of interaction is unity (100\%).

\end{document}
